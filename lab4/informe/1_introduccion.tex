\section{Introducción}
En este laboratorio se busca desarrollar un sismógrafo básico utilizando la placa STM32F429 Discovery Kit y la biblioteca libopencm3, aprovechando la capacidad del giroscopio para detectar movimientos en los ejes X, Y, Z. Este proyecto representa una aplicación práctica del uso de microcontroladores en sistemas de monitoreo sísmico, permitiendo adquirir habilidades en la integración de sensores, manejo de periféricos y transmisión de datos en un entorno de Internet de las Cosas (IoT).
El diseño del sismógrafo se enfoca en capturar las variaciones de los ejes del giroscopio, indicando potenciales eventos sísmicos. A través de un sistema de comunicación USART/USB, se habilitará la transmisión de los datos a una computadora para posteriormente transmitir los datos a una plataforma IoT. Además, se integrará un sistema de monitoreo del nivel de batería, crucial para asegurar la operación continua del dispositivo.
El principal objetivo de este laboratorio es diseñar e implementar un sistema de monitoreo sísmico básico utilizando la placa STM32F429 Discovery Kit, con la finalidad de capturar movimientos en los ejes X, Y, Z mediante el giroscopio incorporado, simulando las funciones de un sismógrafo. Además, se busca proporcionar al usuario la capacidad de habilitar o deshabilitar la transmisión de datos mediante un switch físico, permitiendo controlar la comunicación serial a conveniencia.
El sistema incluirá un LED indicador, que parpadeará cuando la transmisión USART esté habilitada. También se implementará un mecanismo de monitoreo del nivel de batería, en caso de que la tensión descienda por debajo del límite de operación seguro (7V), se encenderá un LED de alarma y se enviará una notificación de batería baja al dashboard IoT en ThingsBoard.
También se planea desplegar en la pantalla LCD los valores del giroscopio en los tres ejes, el nivel de batería, y el estado de la comunicación USART/USB. Finalmente, se desarrollará un script en Python que permitirá transmitir la información del giroscopio y el estado de la batería al dashboard de ThingsBoard, donde los datos podrán ser visualizados de forma intuitiva mediante widgets. 

\vspace{1cm}\\
\textit{Nota: El código fuente se puede encontrar en la rama  ``main" del repositorio, en la carpeta ``lab4". \url{https://github.com/artavias/IE0624/tree/main}}
