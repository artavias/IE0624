\section{Conclusiones}
En este laboratorio, el objetivo era implementar un sistema de monitoreo basado en la lectura de datos del giroscopio en la placa STM32F429 Discovery Kit y su transmisión a la plataforma IoT ThingsBoard. Sin embargo, debido a un fallo en el giroscopio, no fue posible obtener datos reales de los ejes X, Y y Z. Para completar el proyecto, se optó por simular los datos del giroscopio, lo que permitió probar la funcionalidad de adquisición, procesamiento y envío de información a ThingsBoard, cumpliendo así los objetivos de transmisión y visualización en tiempo real.

A pesar de la limitación técnica con el sensor, la simulación de los datos fue útil para verificar el funcionamiento del sistema de comunicación y el correcto envío de la información al dashboard de ThingsBoard. Esta simulación permitió observar en tiempo real cómo los valores del giroscopio (aunque simulados) se desplegaban en la plataforma, confirmando que el sistema era capaz de transmitir datos de manera eficiente y que el dashboard estaba correctamente configurado para recibir y visualizar la información.

La transmisión de datos vía USART se configuró para enviar la información simulada del giroscopio y el estado de la batería. La comunicación mediante USART fue exitosa, ya que el sistema transmitió datos a ThingsBoard en tiempo real, donde se visualizaron correctamente en el dashboard. Adicionalmente, se implementó un control de habilitación mediante un switch físico, lo cual facilitó la activación y desactivación de la transmisión. 

Para mejorar futuros laboratorios, se recomienda comenzar con una verificación previa de los componentes, asegurando que el giroscopio y otros sensores funcionen adecuadamente antes de iniciar la implementación. Ante posibles fallas, sería útil contar con un sistema de simulación de datos automatizado que permita evaluar el rendimiento del sistema bajo condiciones realistas.
